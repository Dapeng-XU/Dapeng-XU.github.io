% $Header$

\documentclass{beamer}

% This file is a solution template for:

% - Talk at a conference/colloquium.
% - Talk length is about 20min.
% - Style is ornate.



% Copyright 2004 by Till Tantau <tantau@users.sourceforge.net>.
%
% In principle, this file can be redistributed and/or modified under
% the terms of the GNU Public License, version 2.
%
% However, this file is supposed to be a template to be modified
% for your own needs. For this reason, if you use this file as a
% template and not specifically distribute it as part of a another
% package/program, I grant the extra permission to freely copy and
% modify this file as you see fit and even to delete this copyright
% notice.


\mode<presentation>
{
  \usetheme{Warsaw}
  % or ...

  \setbeamercovered{transparent}
  % or whatever (possibly just delete it)
}


\usepackage[english]{babel}
% or whatever

\usepackage[latin1]{inputenc}
% or whatever

\usepackage{times}
\usepackage[T1]{fontenc}
% Or whatever. Note that the encoding and the font should match. If T1
% does not look nice, try deleting the line with the fontenc.




\usepackage{listings}
\usepackage{color}

\definecolor{dkgreen}{rgb}{0,0.6,0}
\definecolor{gray}{rgb}{0.5,0.5,0.5}
\definecolor{mauve}{rgb}{0.58,0,0.82}

\lstset{frame=tb,
  language=Pascal,
  aboveskip=3mm,
  belowskip=3mm,
  breaklines=true,
  breakatwhitespace=true,
  basicstyle={\footnotesize\ttfamily},
  showstringspaces=false,
  columns=flexible,
  numbers=left,
  numberstyle=\tiny\color{gray},
  keywordstyle=\color{blue},
  commentstyle=\color{dkgreen},
  stringstyle=\color{mauve},
  tabsize=4
}





\title[Path-based DFS] % (optional, use only with long paper titles)
{Path-based depth-first search for strong and biconnected components}

\author[Harold N. Gabow] % (optional, use only with lots of authors)
{\textbf{Author of the paper: Harold N. Gabow}\newline\newline
 {Reported by: T.T. Liu \and D.P. Xu \and B.Y. Chen } }
% - Give the names in the same order as the appear in the paper.
% - Use the \inst{?} command only if the authors have different
%   affiliation.

\date{\scriptsize{\today} }

\subject{Algorithms}
% This is only inserted into the PDF information catalog. Can be left
% out.



% If you have a file called "university-logo-filename.xxx", where xxx
% is a graphic format that can be processed by latex or pdflatex,
% resp., then you can add a logo as follows:

\pgfdeclareimage[height=0.75cm]{university-logo}{sdulogo}
\logo{\pgfuseimage{university-logo}}



% Delete this, if you do not want the table of contents to pop up at
% the beginning of each subsection:
\AtBeginSubsection[]
{
  \begin{frame}<beamer>{Outline}
    \tableofcontents[currentsection,currentsubsection]
  \end{frame}
}


% If you wish to uncover everything in a step-wise fashion, uncomment
% the following command:

%\beamerdefaultoverlayspecification{<+->}


\begin{document}





\defverbatim[colored]\mycodea{%
\begin{lstlisting}[frame=single, emphstyle={\color{blue}}]
H = G;
while H still has a vertex v
	start a new path P = (v);
	while P is not empty
		if the last vertex of P has an edge (v_k, w)
			if w belongs to P
				find v_i in P, which w and v_i are identical;
				contract the cycle v_i, v_(i+1), ... , v_k, both in H and in P;
			else
				add w to P, as the new last vertex of P;
			end if
\end{lstlisting}
}

\defverbatim[colored]\mycodeb{%
\begin{lstlisting}[frame=single, emphstyle={\color{blue}}, firstnumber=last]
		else
			output v_k as a vertex of the strong component graph;
			// v_k may be a set of multiple vetices in the original graph
		end if
	end
end
\end{lstlisting}
}





\begin{frame}
  \titlepage
\end{frame}

\begin{frame}{Outline}
  \tableofcontents
  % You might wish to add the option [pausesections]
\end{frame}


% Structuring a talk is a difficult task and the following structure
% may not be suitable. Here are some rules that apply for this
% solution:

% - Exactly two or three sections (other than the summary).
% - At *most* three subsections per section.
% - Talk about 30s to 2min per frame. So there should be between about
%   15 and 30 frames, all told.

% - A conference audience is likely to know very little of what you
%   are going to talk about. So *simplify*!
% - In a 20min talk, getting the main ideas across is hard
%   enough. Leave out details, even if it means being less precise than
%   you think necessary.
% - If you omit details that are vital to the proof/implementation,
%   just say so once. Everybody will be happy with that.

\section{Introduction}

\begin{frame}{Several Questions}
	\begin{itemize}
		\item
		One-pass or two-pass?
		\item
		LOWPOINT?
		\item
		Ear decomposition?
		\item
		Compele version?
		\item
		Robbin's Theorem?
	\end{itemize}
\end{frame}

\section{Strong Components}

\begin{frame}{Questions}
	\begin{itemize}
		\item
		"Equivalently the strong component graph is the acyclic digraph,
		formed by contracting vertices of G, that has an many vertices as possible."
		What is the meanings of this sentence?
	\end{itemize}
\end{frame}

\subsection{Purdom and Munro's high-level algorithm}

\begin{frame}{Pseudo-Code}%[fragile]
	\mycodea
\end{frame}

\begin{frame}{Pseudo-Code (Continue.)}%[fragile]
	\mycodeb
\end{frame}

\subsection{Contribution}

\begin{frame}{His Contribution}%{Subtitles are optional.}
  % - A title should summarize the slide in an understandable fashion
  %   for anyone how does not follow everything on the slide itself.

  \begin{itemize}
  \item
    He gave a simple list-based implementation that achieves linear time.
  \end{itemize}
\end{frame}




\section*{Summary}

\begin{frame}{Summary}

  % Keep the summary *very short*.
  \begin{itemize}
  \item
    The \alert{first main message} of your talk in one or two lines.
  \item
    The \alert{second main message} of your talk in one or two lines.
  \item
    Perhaps a \alert{third message}, but not more than that.
  \end{itemize}

  % The following outlook is optional.
  \vskip0pt plus.5fill
  \begin{itemize}
  \item
    Outlook
    \begin{itemize}
    \item
      Something you haven't solved.
    \item
      Something else you haven't solved.
    \end{itemize}
  \end{itemize}
\end{frame}



% All of the following is optional and typically not needed.
\appendix
\section<presentation>*{\appendixname}
\subsection<presentation>*{For Further Reading}

\begin{frame}[allowframebreaks]
  \frametitle<presentation>{For Further Reading}

  \begin{thebibliography}{10}

  \beamertemplatebookbibitems
  % Start with overview books.

  \bibitem{Author1990}
    A.~Author.
    \newblock {\em Handbook of Everything}.
    \newblock Some Press, 1990.


  \beamertemplatearticlebibitems
  % Followed by interesting articles. Keep the list short.

  \bibitem{Someone2000}
    S.~Someone.
    \newblock On this and that.
    \newblock {\em Journal of This and That}, 2(1):50--100,
    2000.
  \end{thebibliography}
\end{frame}

\end{document}


