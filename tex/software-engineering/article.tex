\documentclass[14pt, letterpaper, UTF8, fontset=windowsnew, heading=true]{article}
\usepackage[UTF8]{ctex}
\usepackage{graphicx}
\usepackage{amsmath}
\usepackage{amssymb}

%opening
\title{软件工程复习笔记}
\author{徐大鹏}

\begin{document}

\maketitle

软件工程大水课!一定要考好!求保佑!

\section{软件工程概述}

\subsection{复习提纲里的考点}

\begin{enumerate}
	\item SE的定义、目的、方法及作用(P2 / P16) \\
	定义是什么?方法呢?作用呢?不知道。
	\begin{itemize}
		\item 章前简介:我们的\emph{最终目标}是,生产出高质量软件,进而找到解决方案,并考虑那些对质量有影响的特性。
		\item 1.2节:要写出健壮的、易于理解和维护的并且能以最高效的方式完成工作的代码,必须具备专业软件工程师的技巧和洞察力。因此\emph{软件工程的目标}就是设计和开发高质量软件。
		\item 1.1.2节:\emph{软件工程师的角色}:软件工程师的精力集中于将计算机作为求解问题的工具,而不是研究硬件设计或者算法的理论证明。
	\end{itemize}
	
%	\item (不是考点) 开发模式(paradigm) (P4) \\
%	在1.1.1节:
%	\begin{description}
%		\item[技术] 是产生某些结果的形式化过程。
%		\item[工具] 是用更好的方式完成某件事情的设备或自动化系统。
%		\item[过程] 把工具和技术结合起来,共同生产特定产品。
%		\item[范型(paradigm)] 表示构造软件的特定方法或哲学。
%	\end{description}
%	软件工程师使用工具、技术和范型来提高\emph{软件产品的质量}。
	
	\item
	说明\textbf{错误}、\textbf{缺陷}、\textbf{失败}的含义与联系。(请举例说明)( 6 页)( 44 页习题 3) \\
	当人们在进行软件开发活动的过程中出错(\emph{错误})时,就会出现\emph{故障}。
	\emph{失效}是指系统违背了它应有的行为。
	故障是系统的内部视图,是从开发人员的角度看待系统;失效是系统的外部视图,是从用户的角度。见图\ref{fig:errorfailure}。
	\begin{figure}[h]
		\centering
		\includegraphics[width=0.91\textwidth]{f1-4.PNG}
		\caption{错误和失效}
		\label{fig:errorfailure}
	\end{figure}
	
	\item 软件质量应从哪几个方面来衡量?(P9 -- P12)
	\begin{description}
		\item[产品的质量] 用户角度:易于学习、易于使用;故障的数目少,故障类型都是次要的(次要的、主要的、灾难性的)。设计和编写代码的人员、维护该程序的人员:考虑产品的内部特性,把故障的数目和类型看做产品质量的证据。
		\item[过程的质量] 只要有活动出了差错,产品的质量就会受到影响。提出问题:What? When? Where? How?
		\item[商业环境背景下的质量] 提供的产品和服务。
	\end{description}

%	\item (不是考点) 软件系统的组成(P16) \\
%	第1.5.1节,系统的要素:
%	\begin{description}
%		\item[活动和对象] \emph{活动}是发生在系统中的某些事情,通常描述为由某个触发器引起的事件(事件驱动的),活动通过改变某一特性将一个事物转变成为另一个事物(活动的概念、活动的结果)。活动中涉及的要素称为\emph{对象}或\emph{实体}。通常,这些对象以某种方式相互联系。
%		\item[关系和系统边界] 把\emph{系统}定义为一组事物的集合:一组\underline{实体}、一组\underline{活动}、实体和活动之间\underline{关系}的描述以及系统\underline{边界}的定义。\emph{边界}就是系统包含什么和不包含什么的一个区分。
%	\end{description}

	\item 现代软件工程大致包含的几个阶段及各个阶段文档(P23--P24) \\
	1.6.2节。这个问题也是软件过程由哪几个主要部分组成的答案。
	\begin{description}
		\item[需求分析和定义] 与客户会面以确定需求,这些需求是对系统的描述。
		\item[系统设计] 系统设计告诉客户,从客户的角度看,系统会是什么样的。然后客户要对设计进行评审。当设计得到批准之后,整个系统设计将被用来生成其中单个程序的设计。
		\item[程序设计] 
		\item[编写程序]
		\item[单元测试] 链接之前作为单独的代码段进行测试。
		\item[集成测试] 将模块组合到一起,确保他们能够正确运行。
		\item[系统测试] 对整个系统的测试,用于确保起初指定的功能和交互得以实现。
		\item[系统交付] 
		\item[维护] 出现任何问题,或者需求发生变化时。
	\end{description}

%	\item 
%	(不是考点) 使现代软件工程发生变化的七个关键因素(P28--P29)
%	\\ 1.8.1节:图\ref{fig:sevenreasons}
%	\begin{figure}[h]
%		\centering
%		\includegraphics[width=0.91\textwidth]{c1-8-2.PNG}
%		\caption{使发生变化的七个关键因素}
%		\label{fig:sevenreasons}
%	\end{figure}

	\item
	什么是抽象?(P30)\\
	1.8.2节。\emph{抽象}是在某种概括层次上对问题的描述,使得我们能够集中于问题的关键方面而不会陷入细节。
	
	\item
	什么是软件过程?软件过程的重要性是什么?包含几个阶段?(P32, P45) \\ 
	软件过程的定义和重要性在第二章有相应的问题。包含哪几个阶段在上面的问题中提到过。图\ref{fig:procedure-differences}仅供参考。
	\begin{figure}[h]
		\centering
		\includegraphics[width=0.91\textwidth]{f1-14.PNG}
		\caption{软件过程的差别}
		\label{fig:procedure-differences}
	\end{figure}
	
	\item
	什么是重用等软件工程主要概念?(P34) \\ 1.8.2节。图\ref{fig:multiplexing}
	\begin{figure}[h]
		\centering
		\includegraphics[width=0.91\textwidth]{c1-8-2-6.PNG}
		\caption{复用}
		\label{fig:multiplexing}
	\end{figure}
	
\end{enumerate}

%\subsection{软件工程的重要性}
%第12页终于出现了一点干货:
%\begin{enumerate}
%	\item 新的观点:软件决定计算机系统的价值
%	\item 隐藏在计算机系统背后的困难
%	\begin{itemize}
%		\item 新的观点:软件决定计算机系统的价值
%		\item 
%		非编程问题,用计算机及开发环境本身无法解决。
%		例如:预算、进度、用户需求的优先级处理等等
%		问题。
%	\end{itemize}
%\end{enumerate}

\section{模型化过程和生命周期}

\subsection{提到的考点}

\begin{enumerate}
	\item 什么是软件过程?软件过程的重要性是什么? (P45-46) \\
	2.1节。
	\begin{description}
		\item[重要性] 它强制活动具有一致性和一定的结构。当我们知道如何把事情做好而且希望其他人也能以同样的方式做事时,这些特性就很有用。
		\item[灵活性] 当然,可以在过程模型一致的前提下,不同的开发人员或者开发团队选用不同的技术和工具来完成某个开发过程,这体现了灵活性。
	\end{description}

	\item 瀑布模型及各阶段文档,优缺点? (P49)
	\begin{description}
		\item[优点] 在帮助开发人员布置他们需要做的工作时,瀑布模型是非常有用的。它的\emph{简单性}使得开发人员很容易向不熟悉软件开发的客户做出解释。它\emph{明确地}说明,为了开始下一阶段的开发,那些中间产品是必需的。
		\item[缺点] 瀑布模型并不能反应实际的代码开发方式。除了一些理解非常充分的问题之外,实际上软件是通过大量的迭代进行开发的。
	\end{description}
	
	\item 原型的概念(P51)
	
	\item 论述分阶段开发模型的含义, 其基本分类及特点是什么? (56 页) \\
	2.2.6节,阶段化开发:增量和迭代。 \\
	关键概念:\emph{循环周期}、\emph{产品系统}、\emph{开发系统}。
	基本分类:
	\begin{description}
		\item[增量开发] 
		\item[迭代开发] 
	\end{description}
	
	\item 螺旋模型四个象限的任务及四重循环的含义? (P58)
	\\ \textbf{要做习题!P80--81 页 习题 2, 3。}
	
%	// 在所有的软件开发过程模型中,你认为哪些过程给予你最大的灵活性以应对需求的变更?(81 页习题 11)

	\item 什么是UP,RUP?
	
\end{enumerate}


\section{计划和管理项目}

\subsection{复习提纲里的考点}

\begin{enumerate}
	\item 什么是项目进度?活动?里程碑?(83页)
	\begin{description}
		\item[项目进度] 通过列举项目的各个阶段,把每个阶段分解成\emph{离散的任务或活动},来描述特定项目的软件开发周期。
		\item[活动] 是项目的一部分,在\emph{一段时间}内发生。
		\item[里程碑] 是活动完成的\emph{时刻}。
	\end{description}

	\item 如何计算软件项目活动图的关键路径?冗余时间?最早和最迟开始时间(习题 2, 3) (课堂习题讲解)\\
	关键路径就是最长路径,即每一个节点的时差都为零的路径。冗余时间也就是时差,满足
	$$\text{时差} = \text{可用时间} - \text{真实时间} = \text{最晚开始时间} - \text{最早开始时间}$$
	计算上,要先求出最长路径,然后沿最长路径回溯,找到每一个活动的最早开始时间和最晚开始时间,然后求出每一个活动的时差。
	
	\item 软件团队人员应该具备的能力是什么?(96 页)\\
	3.2.1节。\textbf{不明白}。
	
	\item 软件项目组织的基本结构?(101 页) \\
	主程序员负责制和忘我方法。根据实际情况可以结合这两种极端情况。主程序员负责制的组织结构如何?忘我方法适于那些情况?他们的对比?(对比见图\ref{fig:groupstructure})
	\begin{figure}[h]
		\centering
		\includegraphics[width=0.91\textwidth]{f3-5.PNG}
		\caption{组织结构的对比}
		\label{fig:groupstructure}
	\end{figure}
	
%	\item // 专家估算法的大致含义?(106 页),算式估算法的大致含义?(108 页) \\
%	略。
	
	\item 试述COCOMO模型的三个阶段基本工作原理或含义。(111 页) \\
	3.3.2节。见图\ref{fig:cocomo}
	\begin{figure}[h]
		\centering
		\includegraphics[width=0.91\textwidth]{c3-3-2.PNG}
		\caption{COCOMO模型的三个阶段}
		\label{fig:cocomo}
	\end{figure}
	
	\item 什么是软件风险?主要风险管理活动?有几种降低风险的策略?(P119, P122) \\
	易。
	
	\item 找出图3.23和图3.24(P139)的关键路径。\\
	章末必做练习题。
\end{enumerate}

\section{获取需求}

\subsection{提到的考点}

\begin{enumerate}
	\item 需求的含义是什么?(143 页) \\
	需求就是对期望的行为的表达。需求指定客户想要什么行为,而不是如何实现这些行为。
	\item 需求作为一个工程,其确定需求的过程是什么?(144 页 图 4.1) \\图\ref{fig:collectrequirements}
	\begin{figure}[h]
		\centering
		\includegraphics[width=0.91\textwidth]{f4-1.PNG}
		\caption{获取需求的过程}
		\label{fig:collectrequirements}
	\end{figure}
	\item 举例说明获取需求时,若有冲突发生时,如何考虑根据优先级进行需求分类。(152 页) \\ 4.3.1节。
%	\item // 如何使需求变得可测试?(151-152 页, sidebar4.4)
	\item 需求文档分为哪两类?(153 页) \\ 4.3.2节。
	\item 什么是功能性需求和非功能性需求/质量需求?设计约束?过程约束? (149 页)
	\\ 4.3节开头。
%	\item // 需求的特性?(正确性、一致性、完整性)(155 页)
	\item 知道DFD图的构成及画法(172 页) \\ 4.5.8节
	\item 在需求原型化方面,什么是抛弃型原型?什么是演化型原型?(192--193 页) \\
	4.7节末尾。
%	\item // 用DFD图简单描述ATM机的工作原理(主要功能和数据流)(220 页习题 7)
\end{enumerate}

\section{设计体系结构}

\begin{enumerate}
	\item 什么是软件体系结构?设计模式?设计公约?设计?概念设计?技术设计?(223-224
	页) \\
	5.1节开头,5.1.1节。\textbf{概念设计和技术设计没有找到}。
	
	\item 软件设计过程模型的几个阶段? \\
	跟第四章第二个提到的考点差不多。图\ref{fig:proceduredesignmodel}
	\begin{figure}[h]
		\centering
		\includegraphics[width=0.91\textwidth]{f4-1.PNG}
		\caption{获取需求的过程}
		\label{fig:proceduredesignmodel}
	\end{figure}
	
%	\item // 三种设计层次极其关系?(229 页) \\ 略。
	
	\item 什么是模块化?什么是抽象?(238 页) \\
	
	\item 论述设计用户界面应考虑的问题。(242 页) \\
	
	\item 5.5 节----模块独立性----耦合与内聚的概念及各个层次划分?(248----xxx 页) \\
	
	\item 举例说明耦合与内聚的基本分类。 以及各个分类的含义与特征(284 页习题 4, 5) \\
\end{enumerate}

\section{设计模块}

\begin{enumerate}
%	\item // 什么是面向对象?(286 页) OO 有几个基本特征? 如何使用高级语言实现这些基本特征? 了解并使用高级语言的 OO 基本编程方法和技巧。(286-291) \\
%	略。

	\item 什么是设计模式? \\
	见设计体系结构。
	
	\item OO设计的基本原则?
	
	OO 开发有何优势?(291 页)
	OO 开发过程有几个步骤?(292 页)
	熟悉用例图的组成和画法,用例的几个要素的含义,掌握用例图的实例解析方法(294
	页)
	用例图、类图等对面向对象的项目开发的意义是什么?
	熟悉类图中各个类之间的基本关系分类(303-305)
	熟悉类图等的组成和画法(300-308 页)
	知道 UML 其他图示结构的基本用途。
\end{enumerate}


//为什么说编码工作是纷繁复杂甚至令人气馁?(337 页)
//一般性的编程原则应该从哪三个方面考虑?(340-344 页)
//论述编码阶段实现某种算法时所涉及的问题。(342 页)
在编写程序内部文档时,除了 HCB 外,还应添加什么注释信息?(352-354 页)
什么是极限编程(XP)? 以及派对编程? (357 页)


// 产生软件缺陷的原因?(365 页)
// 将软件缺陷进行分类的理由?(367 页)
几种主要的缺陷类型?(367-368 页)
什么是正交缺陷分类?(369 页)
测试的各个阶段及其任务?(372 页图 8.3)
// 测试的态度问题?(为什么要独立设置测试团队?)(373 页)
掌握测试的方法----黑盒、白盒的概念?(374)
什么是单元测试? 什么是走查和检查?(376 页)
黑盒白盒方法各自的分类? 测试用例的设计和给出方法(结合补充材料)
黑盒白盒方法的分类,各种覆盖方法等。(课件和补充课件)
如何面对一个命题,设计和给出测试用例的问题。(课件)
------课堂练习的测试题目和讲解内容
集成测试及其主要方法的分类?(390-392)(驱动,桩的概念)
// 传统测试和 OO 测试有何不同? OO 测试有何困难? (398-399 页)
// 测试计划涉及的几个步骤?(400 页) (了解)


系统测试的主要步骤及各自含义?(420 页, 图 9.2)
什么是系统配置? 软件配置管理? // 基线? (423 页)(或见课件)
// 什么是回归测试?(425 页)
功能测试的含义极其作用?(430 页)
功能测试的基本指导原则? (431)
性能测试的含义与作用?(436 页)
性能测试的主要分类?(436 页)
// 什么是可靠性、可用性和可维护性?(438 页)
确认测试, 确认测试分类?(基准测试和引导测试)(447-448 页)
什么是 alpha 测试? β测试?(448 页)
// 什么是安装测试?(450 页)

\end{document}
