\documentclass[14pt, letterpaper, UTF8, fontset=windowsnew, heading=true]{article}
\usepackage[UTF8]{ctex}
\usepackage{graphicx}
\usepackage{amsmath}
\usepackage{amssymb}

%opening
\title{软件工程复习笔记}
\author{徐大鹏}

\begin{document}

\maketitle

软件工程大水课!一定要考好!求保佑!

\section{PPT1:为什么要软件工程?}

情怀。
\par
情怀。
\par
情怀。


\subsection{复习提纲里的考点}

\begin{enumerate}
	\item SE的定义、目的、方法及作用(P2 / P16)
	\begin{itemize}
		\item 章前简介:我们的\emph{最终目标}是,生产出高质量软件,进而找到解决方案,并考虑那些对质量有影响的特性。
		\item 1.2节:要写出健壮的、易于理解和维护的并且能以最高效的方式完成工作的代码,必须具备专业软件工程师的技巧和洞察力。因此\emph{软件工程的目标}就是设计和开发高质量软件。
		\item 1.1.2节:\emph{软件工程师的角色}:软件工程师的精力集中于将计算机作为求解问题的工具,而不是研究硬件设计或者算法的理论证明。
	\end{itemize}
	
	\item 开发模式(paradigm) (P4) \\
	在1.1.1节:
	\begin{description}
		\item[技术] 是产生某些结果的形式化过程。
		\item[工具] 是用更好的方式完成某件事情的设备或自动化系统。
		\item[过程] 把工具和技术结合起来,共同生产特定产品。
		\item[范型(paradigm)] 表示构造软件的特定方法或哲学。
	\end{description}
	软件工程师使用工具、技术和范型来提高\emph{软件产品的质量}。
	
	\item
	说明\textbf{错误}、\textbf{缺陷}、\textbf{失败}的含义与联系。(请举例说明)( 6 页)( 44 页习题 3) \\
	当人们在进行软件开发活动的过程中出错(\emph{错误})时,就会出现\emph{故障}。
	\emph{失效}是指系统违背了它应有的行为。
	故障是系统的内部视图,是从开发人员的角度看待系统;失效是系统的外部视图,是从用户的角度。
	\begin{figure}[h]
		\centering
		\includegraphics[width=0.8\textwidth]{f1-4.PNG}
	\end{figure}
	
	\item 软件质量应从哪几个方面来衡量?(P9 -- P12)
	\begin{description}
		\item[产品的质量] 用户角度:易于学习、易于使用;故障的数目少,故障类型都是次要的(次要的、主要的、灾难性的)。设计和编写代码的人员、维护该程序的人员:考虑产品的内部特性,把故障的数目和类型看做产品质量的证据。
		\item[过程的质量] 只要有活动出了差错,产品的质量就会受到影响。提出问题:What? When? Where? How?
		\item[商业环境背景下的质量] 提供的产品和服务。
	\end{description}

	\item 软件系统的组成(P16) \\
	第1.5.1节,系统的要素:
	\begin{description}
		\item[活动和对象] \emph{活动}是发生在系统中的某些事情,通常描述为由某个触发器引起的事件(事件驱动的),活动通过改变某一特性将一个事物转变成为另一个事物(活动的概念、活动的结果)。活动中涉及的要素称为\emph{对象}或\emph{实体}。通常,这些对象以某种方式相互联系。
		\item[关系和系统边界] 把\emph{系统}定义为一组事物的集合:一组\underline{实体}、一组\underline{活动}、实体和活动之间\underline{关系}的描述以及系统\underline{边界}的定义。\emph{边界}就是系统包含什么和不包含什么的一个区分。
	\end{description}

	\item 现代软件工程大致包含的几个阶段及各个阶段文档(P23--P24) \\
	1.6.2:构建系统
	\begin{description}
		\item[需求分析和定义] 与客户会面以确定需求,这些需求是对系统的描述。
		\item[系统设计] 系统设计告诉客户,从客户的角度看,系统会是什么样的。然后客户要对设计进行评审。当设计得到批准之后,整个系统设计将被用来生成其中单个程序的设计。
		\item[程序设计] 
		\item[编写程序]
		\item[单元测试] 链接之前作为单独的代码段进行测试。
		\item[集成测试] 将模块组合到一起,确保他们能够正确运行。
		\item[系统测试] 对整个系统的测试,用于确保起初指定的功能和交互得以实现。
		\item[系统交付] 
		\item[维护] 出现任何问题,或者需求发生变化时。
	\end{description}

	\item 
	使现代软件工程发生变化的七个关键因素(P28--P29)
	\\ 1.8.1节:
	\begin{figure}[h]
		\centering
		\includegraphics[width=0.8\textwidth]{c1-8-2.PNG}
	\end{figure}

	\item
	什么是抽象?( 30 页)
	什么是软件过程?软件过程的重要性是什么?包含几个阶段?( 32 页)( 45 页)
	什么是重用等软件工程主要概念? ( 34 页)
\end{enumerate}

\subsection{软件工程的重要性}
第12页终于出现了一点干货:
\begin{enumerate}
	\item 新的观点:软件决定计算机系统的价值
	\item 隐藏在计算机系统背后的困难
	\begin{itemize}
		\item 新的观点:软件决定计算机系统的价值
		\item 
		非编程问题,用计算机及开发环境本身无法解决。
		例如:预算、进度、用户需求的优先级处理等等
		问题。
	\end{itemize}
\end{enumerate}


\end{document}
